\documentclass[12pt,a4paper,twocolumn,twoside]{jsik}
\usepackage[dvips]{graphicx}

% bhline: hlineを太くする
\makeatletter
\def\bhline{%
%\noalign{\ifnum0=`}\fi\hrule \@height 1.2pt \futurelet
\noalign{\ifnum0=`}\fi\hrule \@height 0.8pt \futurelet
\reserved@a\@xhline}
\makeatother

\newcommand{\TitleJP}{論文題目}
\newcommand{\TitleEN}{Title}

% メタデータ記述
\usepackage{pxjahyper}
\hypersetup{
pdftitle={\TitleJP},
pdfauthor={}
}

\begin{document}

% ページ番号表示の設定.投稿時には有効にすること
% \pagestyle{empty}
% \thispagestyle{empty}


\articletype{研究論文}
% 最終更新日時を表示させたい場合は以下を使う
% \articletype{研究論文 (\textcolor{orange}{最終更新: \the\year 年\the\month 月\the\day 日\the\hour 時 \the\minute 分})}

\jtitle{\TitleJP}
\etitle{\TitleEN}

\jauthor{吉川 次郎$^{\dag*}$}
\eauthor{Jiro KIKKAWA$^{\dag*}$}
\affiliation{
$^{\dag}$ 筑波大学大学院 図書館情報メディア研究科 \\
Graduate School of Library, Information and Media Studies, University of Tsukuba \\
〒305-8550 茨城県つくば市春日1-2 \\
E-mail: jiro@slis.tsukuba.ac.jp \\
$^{*}$ 連絡先著者 Corresponding Author
}

\jabstract{アブストラクト}

\eabstract{Abstract}

\keywords{
Wikipedia,Digital Object Identifier (DOI),学術情報流通 \\
Keywords: \hspace{2.0mm} Wikipedia, \ Digital Object Identifier (DOI), \ Scholarly Communication
}

\maketitle

\subfile{sec1_Introduction}
% \subfile{sec2_Related_Works}
% \subfile{sec3_Methods}
% \subfile{sec4_Results}
% \subfile{sec5_Discussion}
% \subfile{sec6_Conclusion}

\bibliographystyle{jsik}
\bibliography{Reference.bib}

\end{document}
